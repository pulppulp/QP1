\documentclass[12pt, oneside]{article} 
\usepackage[margin=1in]{geometry}   
\usepackage{enumitem}
\geometry{a4paper}    
\usepackage{graphicx}	
\usepackage{tikz}
\usepackage{tikz-qtree}
\usepackage{amssymb}
\usepackage{wrapfig}
\usepackage{tabularx}
\usepackage[T1]{fontenc}
%\usepackage{ulem}
\usepackage{fancyhdr,natbib}
\usepackage{multicol}
\setlength{\columnsep}{-10mm}
\setlength{\bibsep}{0.0pt}
%
\usepackage{float}
\newcolumntype{C}[1]{>{\centering\arraybackslash}p{#1}}
\usepackage{multirow}
\renewcommand*{\familydefault}{\rmdefault}
\usepackage{tipa}
\usepackage{wrapfig}
\usepackage{linguex}
\renewcommand{\firstrefdash}{}
\usepackage[utf8]{inputenc}
\usepackage{xeCJK}
\usepackage{xpinyin}
\xpinyin*{汉语拼音示例}
\title{QP1 meeting note: 1}
\date{February 1st 2019}

\begin{document}

\maketitle

\section{Introduction}
Apart from SVO structure in Mandarin Chinese, there are other constructions that could express the same meaning with different word order. 
\ex. \a. \pinyin{Li3Si4 chai1 le0 na4 jian1 fang2zi.} \\
        LiSi demolish PERF that CL house.\\
        'Lisi demolished that house.'
    \b. \textit{(a \pinyin{ba3} construction equivalent): SbaOV} \\ 
    \pinyin{Li3Si4 ba3 na4 jian1 fang2zi chai1 le0.}\\
    LiSi BA that CL house demolish PERF.
    \c. \textit{(a \pinyin{gei3}construction equivalent): SgeiOV} \\
    \pinyin{Li3 Si4 gei3 na4 jian1 fang2zi chai1 le}\\
    LiSi GEI that CL house demolish PERF.
    \d. \textit{(a \pinyin{ba3...gei3...} construction equivalent):SbaOgeiV}\\
    \pinyin{Li3Si4 ba3 na4 jian1 fang2zi gei3 chai1 le0.}\\
    LiSi BA that CL house GEI demolish PERF. (Bender, 2000)

Difference between (1a) vs (1b),(1c),(1d):\\
The \pinyin{ba3} construction (1b) is also known as "disposal" construction (Wang, 1947) that emphasize how the object is being disposed. (1c) and (1d) have the same disposal emphasis and can be seen as variations of (1b).
The semantic meaning of \pinyin{ba3} construction is stated as in (2) (Hsueh, 1989).
\ex. In connection to A, B turns out to be what C describes. (A as the initial NP, B is the NP that comes after \pinyin{ba3} and C is the remainder of the clause.)

In (1), the initial NP "A" is \textit{LiSi}; "B" the NP that comes after \pinyin{ba3} is  \textit{that house}, and the remainder of the clause "C" is \textit{demolished}. Sentence 1(b) means, in connection to \textit{LiSi}, \textit{that house} turns out to be \textit{demolished}.
\\
\newline
Some Questions need to be asked
   \begin{itemize}[noitemsep]
         \item What are the properties of the \pinyin{ba3} construction?
         \item Can all \pinyin{ba3} construction (SbaOV)be transformed into \pinyin{gei3} construction (SgeiOV) and \pinyin{ba3...gei3...} construction (SbaOgeiV)?
         \item Why \pinyin{gei3} construction (SgeiOV) and \pinyin{ba3...gei3...} construction (SbaOgeiV) can function the same as \pinyin{ba3} construction (SbaOV)?
            \begin{itemize}[noitemsep]
                 \item What are properties of \pinyin{gei3} construction?
                 \item What is the interaction between \pinyin{ba3} and \pinyin{gei3}?  
          \end{itemize}
    \end{itemize}
\section{The basic properties of the \pinyin{ba3} construction}
(revisiting analyses in Bender (2000))
\newline
She claimed that there are two types of \pinyin{ba3} construction: those have a SVO equivalent and those don't. However, her example can be transformed into a SVO structure without \pinyin{ba3}.  
\ex. \a. (Bender's example)\\
    \pinyin{La4jiao1 ba3 wo3 chi1 de4 she2tou0 dou la4 ma2 le0.}\\
    Hot pepper BA I eat DE tongue all numb PERF.\\
    i. 'The hot pepper that I ate made my tongue numb.'\\
    ii. 'The hot pepper made me have a numb tongue by way of my eating it.'
    \b. (SVO equivalent)\\
    \pinyin{La3jiao1 chi1 de0 wo3 she2tou0 dou1 ma2 le0.}\\
    Hot pepper eat DE I tongue all numb. 
    \c. (\pinyin{gei3} construction)\\
    ? \pinyin{La4jiao1 gei3 wo3 chi1 de4 she2tou0 dou la4 ma2 le0.}\\
    Hot pepper GEI I eat DE tongue all numb PERF.\\
    \textit{(acceptable in northern dialects)}
    \d. (\pinyin{ba3...gei3...} construction)\\
    \pinyin{La4jiao1 ba3 wo3 gei3 chi1 de4 she2tou0 dou la4 ma2 le0.}\\
    Hot pepper BA I GEI eat DE tongue all numb PERF.\\
 
Then she went on to further review the literature that viewed \pinyin{ba3} as a case maker and as a verb; and provided two trees for each analysis. (Her analyses supported the view that \pinyin{ba3} as a verb). 
\ex. \begin{center}
    \Tree
\end{center}


One problem with her analyses is that the syntactic structure for \pinyin{ba3} as a verb is flat.
\newpage
Some interesting data:
\ex. \pinyin{Hou2zi0 qi2lei4 le0 tu4zi.}\\
Monkey ride-tired PERF rabbit.\\
Ambiguous:\\
'The monkey rode the rabbit tired.'\\
i. 'The monkey rode on the rabbit and as a result, the rabbit became tired.'\\
ii. 'The monkey rode on the rabbit and as a result, the monkey became tired.'
\\ (Mark, 1995)

This sentence is ambiguous in Mandarin Chinese that it can have the two readings: 'Rabbit is tired(5(i))' vs 'Monkey is tired(5(ii))'. Both readings are equally possible. 5(i) reading can be realized through \pinyin{ba3} constructions in (6). 5(ii) reading can be realized through double verb construction (7). (Bender's account for the readings not available in (6) is that "‘the monkey became tired’ does not constitute a description of how the rabbit turns out", which is a semantic explanation.)
\ex. \a. (\pinyin{ba3} construction)\\
    \pinyin{Hou2 zi0 ba3 tu4 zi0 qi2 lei4 le0.}\\
    Monkey BA rabbit ride tired PERF. \\
    'The monkey rode on the rabbit and as a result, the rabbit became tired.'
    \b. (\pinyin{gei3} construction)\\\pinyin{Hou2 zi0 gei3 tu4 zi0 qi2 lei4 le0.}\\
    Monkey GEI rabbit ride tired PERF. 
    \c. (\pinyin{ba3...gei3...}construction)\\
    \pinyin{Hou2 zi0 ba3 tu4 zi0 gei3 qi2 lei4 le0.}\\
    Monkey BA rabbit GEI ride tired PERF. 
    
\ex. (double verb construction)\\
\pinyin{Hou3 zi0 qi2 tu4 zi0 qi2 lei4 le0}.\\
Monkey ride rabbit ride tired PERF. \\ 
'The monkey rode on the rabbit and as a result, the monkey became tired.'

Sentences that can only be expressed in SVO but not SbaVO. 
\ex. \a. \pinyin{wo3 kan4 jian4 le0 lan2 tian1.}\\
I see PERF blue sky.
'I saw/see the blue sky.'
\b. \pinyin{\# wo3 ba3 lan2 tian1 kan4 jian4 le0.}\\
\# I BA blue sky see PERF.(from Zhang 2000) 

\ex. \a. \pinyin{wang3mian3 si3 le0 fu4qin1.}\\
Wangmian die PERF father.\\
'Wangmian's father died.'
\b. \pinyin{*wang3mian3 ba3 fu4qin1 si3 le0.}\\
\text{*}Wangmian BA father die PERF.(from Shen 2006)

\ex. \a. \pinyin{Qiang2 shang4 gua4 le0 yi1 fu2 hua4.}\\
Wall on hang PERF one CL picture.\\
'There's a picture hanging on the wall.'
\b. \pinyin{*Qiang2 shang4 ba3 yi1 fu2 hua4 gua4 le0.}\\
\text{*}Wall on BA one CL picture hang PERF.




To understand the function or part of speech of \pinyin{ba3}, it is essential to know the role of initial NP that comes before \pinyin{ba3} (in relation to the post-\pinyin{ba3} verb).\\
First, it can be the \textbf{subject} + \textbf{agent} of the post-\pinyin{ba3} verb (as in (1) and (11)).
\ex. \a. \textit{(SVO construction)}\\\pinyin{wo3 yi3jing1 mai4 le0 wo3 de0 qi4che1.}\\
I already sell PERF I DE\textsubscript{NP} car.\\
'I already sold my car.'
\b. \textit{(\pinyin{ba3} construction):SbaOV}\\
\pinyin{wo3 yi3jing1 ba3 wo3 de0 qi4che1 mai4 le0.}\\
I already BA I DE\textsubscript{NP} car sell PERF.
\c.\textit{(\pinyin{gei3} construction):SgeiOV}\\ \pinyin{wo3 yi3jing1 gei3 wo3 de0 qi4che1 mai4 le0.}\\
I already GEI I DE\textsubscript{NP} car sell PERF.
\d.\textit{(\pinyin{ba3...gei3...} construction):SbaOgeiV}\\ \pinyin{wo3 yi3jing1 ba3 wo3 de0 qi4che1 gei3 mai4 le0.}\\
I already BA I DE\textsubscript{NP} car GEI sell PERF.
\\(from Li and Thompson (1981))

Second, it can be only \textbf{subject} but not agent (as in (12))
\ex. \a. \textit{SVO construction}\\
\pinyin{Ai1yuan4 de di2sheng1 chui1 de wo3men0 xin1suan1lei4luo4.} \\
Morning DE\textsubscript{NP} flute-voice chui1 DE we heart-distressed-tears-fall.\\
'The morning tune of the flute made us heartbroken and tearful.'
    \b.\textit{(\pinyin{ba3} construction):SbaOV}:
\\
\pinyin{Ai1yuan4 de di2sheng1 ba3 wo3men0 chui1 de xin1suan1lei4luo4.}\\
Morning DE\textsubscript{NP} flute-voice BA we blow DE heart-distressed-tears-fall.
    \c. \textit{(\pinyin{gei3} construction):SgeiOV}:\\ \pinyin{Ai1yuan3 de di2sheng1 gei3 wo3men0 chui1 de xin1suan1lei4luo4.}\\
    Morning DE\textsubscript{NP} flute-voice BA we blow DE heart-distressed-tears-fall.
    \d. \textit{(\pinyin{ba3...gei3...construction}: SbaOgeiV)}\\ \pinyin{Ai1yuan4 de di2sheng1 ba3 wo3men0 gei3 chui1 de xin1suan1lei4luo4.}\\
    Morning DE\textsubscript{NP} flute-voice BA we GEI blow DE heart-distressed-tears-fall.

The \textit{morning tune of the flute} is the subject in the SVO construction, but it is not the agent. The verb in this sentence is \textit{blow}. The relationship between \textit{morning tune of the flute} and \textit{blow} is complicated. 

\newline
Third, the initial NP can be the \textbf{object} in the SVO structure. 

\ex. \a. \textit{SVO construction}: double verb construction\\
\pinyin{ni3 zen3me0 pa4 zhe4 jian4 shi4 pa4cheng2 zhe4 yang4}.\\
You how fear this CL matter fear-become this shape.\\
'How did this matter make you fear like that?'
\b. \textit{(\pinyin{ba3} construction):SbaOV} \\
\pinyin{zhe4 jian4 shi4 zen3me0 ba3 ni3 pa4cheng2 zhe4 yang4.}\\
This CL matter how BA you fear-become this shape.\\
\c. \textit{(\pinyin{gei3} construction): SgeiOV}\\
\pinyin{zhe4 jian4 shi4 zen3me0 gei3 ni3 pa4cheng2 zhe4 yang4.}\\
This CL matter how GEI you fear-become this shape.
\d. \textit{(\pinyin{ba3...gei3...} construction: SbaOgeiV)}\\
\pinyin{zhe4 jian4 shi4 zen3me0 ba3 ni3 gei3 pa4cheng2 zhe4 yang4.}\\
This CL matter how BA you GEI fear-become this shape. 






\end{document}
