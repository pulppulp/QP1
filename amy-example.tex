\documentclass[12pt, oneside]{article}   	% use "amsart" instead of "article" for AMSLaTeX format
\usepackage[margin=1in]{geometry}                		% See geometry.pdf to learn the layout options. There are lots.
\geometry{a4paper}                   		% ... or a4paper or a5paper or ... 
%\geometry{landscape}                		% Activate for for rotated page geometry
%\usepackage[parfill]{parskip}    		% Activate to begin paragraphs with an empty line rather than an indent
\usepackage{graphicx}				% Use pdf, png, jpg, or eps with pdflatex; use eps in DVI mode
								% TeX will automatically convert eps --> pdf in pdflatex		
\usepackage{tikz}
\usepackage{tikz-qtree}
\usepackage{amssymb}
\usepackage{wrapfig}
\usepackage{tabularx}
\usepackage[T1]{fontenc}
%\usepackage{ulem}
\usepackage{fancyhdr,natbib}
\usepackage{multicol}
\setlength{\columnsep}{-10mm}
\setlength{\bibsep}{0.0pt}
%
\newcolumntype{C}[1]{>{\centering\arraybackslash}p{#1}}
\usepackage{multirow}
\renewcommand*{\familydefault}{\rmdefault}
\usepackage{tipa}
\usepackage{wrapfig}
\usepackage{linguex}
\renewcommand{\firstrefdash}{}





%\date{CUNY}							% Activate to display a given date or no date
\pagenumbering{gobble}

\begin{document}
 \vspace*{-15mm}

\begin{center}
\Large
\textbf{The syntax of  b\v{a} ... g\v{ě}i ...  construction in Mandarin Chinese\\
Core Area: Syntax | Advisor: Prof. Bill Haddican }\normalsize\\

%

\end{center}

%% Different approach: This paper presents an analysis of Galician Solidarity clitics as allocutive morphemes, following a
%%  (i) Doubling; (ii) subject-object asymmetry in Addressee blocking; (iii) embedded contexts--must be dissociated from indexical shift.

%% New new--A question that arises is what the relationship of this element is to the element responsible for agreement 

%1. Subject-object
%2. Doubling.
%4. Embedded

%empirical progress.Recent literature has brought considerable progress in formal descriptions of allocutivity 
%\vspace*{-.4cm}


	Verbal argument structure remains one of the most controversial issues in Mandarin syntax. Apart from SVO construction, there are alternative constructions with case markers that could express the same meaning as an SVO structure (illustrated in 1a, 1b and 1c). The bǎ construction and gěi construction are two of the most widely discussed phenomena in the literature. Nonetheless, there is little consensus in the literature on the structure of these forms or the category of bǎ and gěi (e.g. Bender, 2000; Her, 2006; Huang and Mo, 2000).

\ex. \a. Lĭ Sì	 chāi	 	le 	nà 	jiān 	fángzi\\
         Li Si	 demolish     PERF  that	 CL 	house.\\
	‘Li Si demolished that house.’
      \b. (a bǎ construction equivalent )
	Lĭ Sì 	bǎ 	nà 	jiān 	fángzi 	     (gěi) 	chāi 		le. \\
	Li Si	ba	that 	CL	house	     (gei)	demolish	PERF.	\\
      \c.  (a gěi construction equivalent )\\
	Lĭ Sì	gěi 	nà 	jiān 	fángzi      chāi 	le.\\
	Li Si	gei	that 	CL	house	     demolish	PERF.	


	One phenomenon previous studies have overlooked is that it is equally acceptable to insert gěi in sentences with bǎ construction without changing the meaning of the sentence (1b). This paper, instead of treating bǎ construction and gěi construction as independent structures, is going to investigate bǎ…gěi… construction as a whole unit and attempt to offer analysis that would apply to both constructions. 
	The preliminary analysis demonstrates that bǎ and gěi both introduce arguments under a shell structure (shown in figure 1). The head of the shell structure would be a ‘VP’; and bǎ and gěi  are light verb ‘v’ (Hale and Keyser, 1998).

 

\ex.  \begin{tikzpicture}[baseline]\tikzset{level distance=25pt, sibling distance=0pt}
\Tree [.VP [.{} ]  [.V'  [.v ba ]  [.vP  [.Arg1 ]  [.v' [.v gei ] [.vp [.Arg2 ]  [.v-root ]  ] ] ]   ] ]
\end{tikzpicture} 



			Figure 1. Basic structure of bǎ…gěi…construction
Basic on the behavior of bǎ and gěi, argument that introduced by bǎ (Arg1) is more likely to have semantic role of an actor, or causer of the main event; argument introduced by gěi (Arg2) usually take the rest of the thematic roles that are necessary to complete the semantic meaning of the sentence. This paper is going to further provide an analysis of the part of speech of bǎ and gěi, constructions with bǎ and gěi that cannot be converted to a SVO structure, and other evidence in ancient Chinese and language acquisition involve bǎ…gěi…construction. 









%\vspace{2mm}
%\noindent\textbf{References:}
%\vspace*{-2.5mm} 
%\begingroup
%\renewcommand{\section}[2]{}\nolinebreak
%\bibliography{syntax.bib}
%\bibliographystyle{lingua.bst}
%\endgroup


\end{document}